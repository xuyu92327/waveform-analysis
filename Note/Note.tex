% !TEX program = pdflatex
\documentclass{article}
\usepackage{titletoc}
\usepackage{graphicx}
\usepackage{geometry}
\usepackage{booktabs}
\usepackage{indentfirst}
\usepackage{tikz}
\usepackage{float}
\usepackage{subfigure}
\usepackage{cite}
\usepackage{hyperref} 
\usepackage[fleqn]{amsmath}
\usepackage{amssymb}
\usepackage{amsfonts}
\usepackage{mathrsfs}
\usepackage{pdfpages}
\usepackage[justification=centering]{caption}
\usepackage{listings}
\usepackage{authblk}

\hypersetup{citecolor=pureblue}%文献引用颜色
\bibliographystyle{ieeetr}
\setlength{\parindent}{2em}
\titlecontents{section}[0mm]
                        {\vspace{.2\baselineskip}\bfseries}
                        {\thecontentslabel~\hspace{.5em}}
                        {}
                        {\dotfill\contentspage[{\makebox[0pt][r]{\thecontentspage}}]}
                        [\vspace{.1\baselineskip}]

\title{Notes on PMT Waveform Analysis and Ghost Hunter 1}
\author[1]{Dacheng Xu\thanks{xdc17@mails.tsinghua.edu.cn}}
\affil[1]{Department of Engineering Physics, Tsinghua University}
\author[2]{Yiyang Wu\thanks{wuyiyang17@mails.tsinghua.edu.cn}}
\affil[2]{Department of Physics, Tsinghua University}
\date{2020}

\begin{document}
\maketitle

\tableofcontents
\section{Introduction} % (fold)
\label{sec:Introduction}
Traditionally, we only save charge integration and time over threshold of the waveform output of PMT in neutrino experiments. But extracting the information of time and number of photons hitting will provide more detailed information of light transmit model in detector and will finally enhance the performance of event reconstruction, even particle identification. Therefore PMT waveform analysis is necessary when we presume completely usage of PMT output. 

In April 2018, Jinping Neutrino Experiment group proposed a \href{https://mp.weixin.qq.com/s?__biz=MzA4MTAwMzgzOA==&mid=2650872289&idx=2&sn=48145a6598545d201f940e0459de99dd&chksm=846e2db0b319a4a627e902d0d6ed4b9d968225566021342c5935764963f352fbe02db1bdb333&mpshare=1&scene=1&srcid=0307c4HOvK0ChJUcq9blC3ub%23rd}{online data contest} whose topic is PMT waveform analysis. During the contest, Wasserstein distance and Poisson distance are used to assess the performance of algorithm. After contest, we collected several algorithms submitted by participants and we checked, redesigned and tested these algorithms to provide practical data processing flow applied on neutrino experiments where PMT is included. 
% section Introduction (end)
\section{Waveform of PMT} % (fold)
PMT is a device which is highly sensitive to even single photon. Therefore, PMT is widely used in neutrino experiments based on liquid and dark matter experiment. In neutrino experiments, liquid scintillator emits light after excited by candidate particles and charged particles emit Cherenkov light in liquid. 
% section Waveform of PMT (end)
\section{Wasserstein-distance and Poisson-distance} % (fold)
\subsection{Wasserstein distance}
\label{sub:Wasserstein distance}
    \begin{equation}
        W_{p}(\mu ,\nu ):=\left(\inf _{\gamma \in \Gamma (\mu ,\nu )}\int _{M\times M}d(x,y)^{p}\,\mathrm {d} \gamma (x,y)\right)^{1/p}
        \label{eq:w-dist-def}
    \end{equation}
    \begin{figure}[H]
        \centering
            \includegraphics[width=0.95\textwidth]{figures/Wasserstein.pdf}
        \caption{Wasserstein transportation}
        \label{fig:Wasserstein transportation}
    \end{figure}
% subsection Wasserstein distance (end)
\subsection{Poisson distance}
% subsection Poisson distance (end)
% section Wasserstein-distance and Poisson-distance (end)
\section{Algorithm} % (fold)

\subsection{takara}

% subsection takara distance (end)

\subsection{xiaopeip}

% subsection xiaopeip distance (end)

\subsection{mcmc}

% subsection mcmc distance (end)
% section Algorithm (end)
\section{Performance} % (fold)
\section{Conclusion} % (fold)
\end{document}